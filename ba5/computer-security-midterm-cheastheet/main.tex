\documentclass[8pt]{article}
\usepackage[top=0.2in, bottom=0.2in, left=0.1in, right=0.1in]{geometry} % Adjust margins individually
\usepackage{amsmath,amssymb,amsthm}
\usepackage{enumitem}
\usepackage{graphicx} % Required for \includegraphics
\usepackage{ragged2e}
             
\usepackage[absolute,overlay]{textpos}
\usepackage{array} % For better column definitions
\usepackage{booktabs} % For improved table aesthetics
\raggedbottom
\setlist[itemize]{leftmargin=1em}
\newcommand{\dist}[2]{\left\langle #1,\, #2 \right\rangle}
\setlength{\TPHorizModule}{0.1mm} % Set horizontal units to millimeters
\setlength{\TPVertModule}{0.1mm}  % Set vertical units to millimeters
\usepackage{multicol}
\setlength{\parindent}{0pt}
\setlength{\parskip}{5pt plus 1pt minus 1pt}
\usepackage{setspace}
\setstretch{1}
\pagestyle{empty}
\setlength{\parskip}{0pt}      % space between paragraphs
\setlength{\parindent}{0pt}    % optional: remove indentation
\setlist{nosep, topsep=0pt, partopsep=0pt, parsep=0pt, itemsep=0pt}
% Redefine the textblock environment
\let\originaltextblock\textblock
\let\endoriginaltextblock\endtextblock

\renewenvironment{textblock}[2][]{%
    \originaltextblock[#1]{#2}%
    \fcolorbox{red}{white}{%
    \begin{minipage}{#2}%
}{%
    \end{minipage}%
    }%
    \endoriginaltextblock
}


\begin{document}
\begin{titlepage}
	\centering
	\vspace*{1cm}
	{\Huge \textbf{Computer Security - CheatSheet}} \\
	\vspace{20px}
	{\LARGE IN~BA5 - Thomas Bourgeat} \\
	\vspace*{1cm}
	{\Large Notes by Ali EL AZDI} \\
	\vfill

	\begin{justify}

	\end{justify}
	\vspace*{100px}

	{\large October 26th, 2025}
	\vspace*{20px}
\end{titlepage}


\vspace*{-20px}
\noindent
\begin{minipage}[t]{0.49\textwidth}
	\noindent\textbf{CompSec Properties}
	\begin{itemize}
		\item[-] \textbf{Confidentiality}. prevention of unauthorized disclosure of information. \textit{authorized users may read a file}
		\item[-] \textbf{Integrity.} prevention of unauthorized modification of information. \textit{authorized programs may write a file}
		\item[-] \textbf{Availability.} prevention of unauthorized denial of service or access to information and resources.\\ \textit{authorized services can access a file}
		\item[-] \textbf{Authenticity.} assurance that entities (users, systems, or data) are genuine and can be verified as such.
		\item[-] \textbf{Anonymity.} protection of an individual's identity from being disclosed or linked to specific actions or data.
		\item[-] \textbf{Non-repudiation.} assurance that a party in a communication cannot deny the authenticity of their signature or the sending of a message.
	\end{itemize}
	\noindent\textbf{The Adversary.} malicious entity aiming at breaching the security policy and \textbf{will} choose the optimal way to use her ressources to mount an attack that violates the security properties.\\
	\noindent \textbf{Threat Model.} describes the ressources available to the adversary and their capabilities \textit{(has access to internet, but doesn't have access to the internal network of the company.)}\\
	\noindent \textbf{Threat.} Who might attack which assets, using what resources, with what goal, how, and with what probability\\
	\noindent \textbf{Vulnerability.} Specific weakness that could be exploited by adversaries with interest in a lot of different assets \textit{(API is not protected, password appears in plain text\dots)}
\end{minipage}
\hfill
\noindent
\begin{minipage}[t]{0.49\textwidth}
	\noindent \textbf{Harm.} The bad thing that could happen when the \textbf{threat} materializes. \textit{(adversary steals the money, learns my password\dots)}
	\noindent \textbf{Security Policy.} high level description of the security properties that must hold in the system in relation to assets and principals
	\begin{itemize}
		\item[-] \textbf{Assets (objects).} anything with value (data, files, memory) that needs protection.
		\item[-] \textbf{Principals (subjects).} people, computer programs, services
	\end{itemize}
	\noindent \textbf{Security Mechanism.} Technical mechanism used to ensure that the security policy is not violated by an adversary within the threat model, \textbf{we can only prepare for threats we're aware of} \\\textit{(Policy. ensure messages cannot be read by anyone but the sender and the receiver, Mechanism. encrypt the message before sending)}
	\noindent \textbf{Composition of Security Mechanisms}
	\begin{itemize}
		\item[-] \textbf{Defence in depth.} As long as one remains unbroken the Security Policy isn't broken) \textit{(two-factor auth)}
		\item[-] \textbf{Weakest Link.} if anyone fails the Security Policy, it is broken. \textit{(security questions for a lost password, just need to know the answer.\dots)}
	\end{itemize}
	\textit{Humans can be vulnerabilities - phishing attacks, bad use of passwords\dots)}\\
	\noindent \textbf{To show a system is secure. (under a \textit{specific} threat model)}
	\begin{itemize}
		\item[] Attacker - Just one way to violate \textbf{one} security property is enough.
		\item[] Defender - No adversary strategy can violate the security policy.
	\end{itemize}
	\noindent \textbf{Security Argument.} Rigorous argument that the security mechanisms in place are indeed effective in maintaining the security policy subject to the assumptions of the Threat Model.
\end{minipage}\\[5px]
\hline
\vspace{5px}
\noindent \begin{minipage}[htp]{0.49\textwidth}
	\noindent \textbf{Principles of CompSec.}
	\begin{itemize}
		\item[1.] \textbf{Economy of mechanism}\\
		      Keep the security mechanism/implementation design as simple and small as possible. Why ?
		      \begin{itemize}
			      \item[a.] Easier to audit and verify.
			      \item[b.] Testing is not appropriate to evaluate security.
		      \end{itemize}
		      \noindent \textbf{Trusted Computing Base (TCB).}\\
		      Every component of the system on which the security policy relies upo hardware, firmware, software. \\The TCB is trusted to operate correctly for the security policy to hold. $\to$ If something goes wrong in it, the security policy may be violated\\
		      It \textbf{must} be kept small to ease verification (economy of mechanism) and diminish the attack surface

		\item[2.]\textbf{Fail-safe defaults.} Base access decisions on permission rather than exclusion. \textit{(Whitelist, do not blacklist)} \\
		      If something fails, be as secure as it does not fail errors / uncertainty should error on the side of the security policy
		      Do \textbf{not} try to fix on error !
		      \begin{itemize}
			      \item Automated doors: if they cannot close, stay open
			      \item Form input: if no permission to write in X, do not write anywhere
		      \end{itemize}
		\item[3.]  \textbf{Complete mediation.}
		      \textbf{Every} access to every object must be checked for authority
		      A Reference Monitor mediates all actions from subjects on objects and ensures they are according to the policy. Tradeoff time\_to\_check vs. time\_to\_use
		\item[4.] \textbf{Open design} The design should not be secret
		      \begin{itemize}
			      \item Always design as if the enemy knows the system.
			      \item When you design...
			            \begin{itemize}
				            \item	Crypto. Only keep the key secret
				            \item Authentication. Only keep the password secret
				            \item Obfuscation. Only keep the used noise secret
			            \end{itemize}
		      \end{itemize} assuming the thread model can't get a hold of the system is unrealistic (employee corruption, ...)
	\end{itemize}
\end{minipage}
\hfill
\noindent \begin{minipage}[htp]{0.49\textwidth}
	\begin{itemize}
		\item[5.] \textbf{Separation of privilege.}	No single accident, deception, or breach of trust is sufficient to compromise the protected information
		      \begin{itemize}
			      \item \textbf{Privilege.}\
			            A privilege allows a user to perform an action on a computer system that may have security consequences. \textit{(create a file in a directory, access a device, write to a socket for communicating over the internet\dots)}
		      \end{itemize}

		\item[6.] \textbf{Least Privilege.}\\
		      Every program and every user of the system should operate using the least set of privileges necessary to complete the job. \textit{Rights are added as needed, discarded after use. Users should get to know about things if they \textbf{have} to.}

		\item[7.] \textbf{Least Common Mechanism}\\
		      Minimize the amount of mechanism common to more than one user and depended on by all users. Every shared mechanism represents a potential information path between users.
		\item[8.] \textbf{Psychological acceptability}
		      It is essential that the human interface be designed for ease of use, so that users routinely and automatically apply the protection mechanisms correctly. \textit{(hide complexity, keep ressources as accessible as before,mental model of users must match security policy/mechanisms, cultural acceptability\dots)}
		\item[9.] \textbf{Work Factor}\\
		      Compare the cost of breaking the mechanism with the resources of a potential attacker. \textit{(cost of compromising insiders, cost of finding a bug, monetization\dots)}
		\item[10.] \textbf{Compromise recording}\\
		      Reliably record that a compromise of information has occurred [...] in place of more elaborate mechanisms that completely prevent loss.
		      (\textit{keep tamper-evidence logs, what if you cannot recover them? (confidentiality), how to keep integrity ? (Blockchain), logs may be a vulnerability ? (Privacy), logging the log ? (Availability)}\ldots)
	\end{itemize}
\end{minipage}

\newpage
\vspace*{-20px}
\begin{minipage}[htp]{0.49\textwidth}
	\textbf{Access Control.}
	Security mechanism that ensures that all accesses and actions on objects by principals are within the security policy.\\
	\textbf{no} \textit{chicken soup (checks everywhere in code), use a reference monitor, module used all over code checking for \textbf{subjects} and \textbf{actions}}\\
	\textbf{Discretionary Access Control (DAC).}
	Object owners assign permissions, ownership of resources. \textit{Linux, Social Networks}\\
	\textbf{Mandatory Access Control (MAC).} Central security policy assigns permissions, usually for organizations with need for central control. \textit{Military, Hospital Environments, Banking}.\\
	\textbf{Access Control Matrix.} \textbf{abstract representation} of all permitted triplets of (subject, object, access right) within a system.
	\includegraphics[width=1\textwidth]{assets/access-control-matrix.png}
	\textit{Complexity.} $O(f\cdot u)$\\
	\textbf{Access Control List (ACL).} associate permissions to objects, stores permissions close to the resource.\\
	\textit{file1: {(Alice,read/write)}},
	\textit{file2: {(Bob, read/write)}},\\
	\textit{file3: {(Alice,read),(Bob,read/write)}}\\
	$\oplus$ easy to determine who can access a resource and to revoke rights by resource.\\
	$\ominus$ difficult to check all users rights, to remove all permissions for a user. it's also difficult to delegate perms.\\
	\textbf{Role Based Access Control (RBAC).} access granted based on user roles and predefined permissions. systems have too many subjects (that come and go) $\rightarrow$ large dynamic ACLs.\\
	Subjects are often similar to each other and get assigned the same rights.\\
	1. Assign permissions to roles, \quad 2. Assign roles to subjects\\
	3. Subjects select a role, they have the permissions of the active role\\[0.1em]
	Problems: role explosion (temptation to create fine grained roles), limited expressiveness, difficult to implement separation of privilege.\\
	\textbf{Group Based Access Control (GBAC).} access granted based on user group membership. \textbf{exactly like RBAC but instead of roles, permissions are assigned to groups.} groups are (typically) broader, less specialized, often representing organizational units rather than specific functions.
	\begin{itemize}
		\item[1.] Assign permissions to groups
		\item[2.] Assign subjects to groups
		\item[3.] Subjects have the combined permissions of all their groups
	\end{itemize}
	Problems: coarse granularity, overlapping group memberships, inconsistent permissions, difficult to manage if users belong to many groups.\\
	\textit{In case of \textbf{Negative Permissions} check negative permissions first before group, think of system crashes before checking negatives.}\\[2px]
	\noindent \textbf{Capabilities.} associate permissions to \textit{subjects}, stores permissions close to the user/process.\\
	\textit{Alice: \{(file1, read/write), (file3, read)\}}\\
	\textit{Bob: \{(file2, read/write), (file3, read/write)\}}\\
	$\oplus$ easy to determine all permissions of a user and to delegate rights by subject.\\
	$\ominus$ difficult to determine who can access a given resource, and to revoke rights by resource.\\
\end{minipage}
\hfill
\begin{minipage}[htp]{0.49\textwidth}
	\textbf{Ambient Authority.} an action succeeds if the subject only specifies the \textit{operation} and the \textit{object name}, not the specific authority used. $\ominus$ leads to accidental misuse of authority, programs may act with more rights than intended. $\rightarrow$ Confused Deputy Problem.\\
	\textbf{Confused Deputy Problem.} when a program (deputy) is tricked into misusing its authority on behalf of another subject. \textit{students ask a professor (who has access) to open a staff-only room, the professor unintentionally bypasses policy.}\\
	\textit{Compiler example:} Alice runs compiler(input, bill) $\Rightarrow$ compiler (with write access to bill) overwrites billing file.\\
	$\Rightarrow$ Alice uses compiler’s authority to modify bill indirectly.\\
	$\ominus$ ambient authority allows unintended privilege use.\\
	$\oplus$ \textbf{Solutions:}\\
	1. Restrict privileged process access.\\
	2. Make privileged process check user’s authorization.\\
	3. Use \textbf{Capabilities} to explicitly delegate rights.\\

\end{minipage}
\end{document}

