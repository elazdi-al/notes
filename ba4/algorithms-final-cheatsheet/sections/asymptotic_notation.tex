\fbox{
  \begin{minipage}[t]{1\linewidth}
  \textbf{Big-O}\\
  If $\exists c>0$ and $\exists n_0>0$, $0\le f(n)\le c\cdot g(n)$ $\forall n\ge n_0$, $f(n)=O(g(n))$.\\
  \textbf{Big-Omega}\\
  If $\exists c>0$ and $\exists n_0>0$, $0\le c\cdot g(n)\le f(n)$ $\forall n\ge n_0$, $f(n)=\Omega(g(n))$.\\
  \textbf{Big-Theta}\\
  If $f(n)=O(g(n))$ and $f(n)=\Omega(g(n))$, $f(n)=\Theta(g(n))$.\\
  \textbf{Little-o}\\
  If $\forall c>0$ $\exists n_0>0$, $0\le f(n)< c\cdot g(n)$ $\forall n\ge n_0$, $f(n)=o(g(n))$.\\
  \textbf{Relations}\\
  $f(n)=o(g(n)) \implies f(n)=O(g(n))$\\
  \textbf{Comparison of Common Functions (Ascending Order)}\\
  $O(1) \ll O\bigl((\log n)^c\bigr) \ll O\bigl(n^c\bigr)_{0 < c < 1} \ll O(n) \ll O\bigl(n \log n\bigr) \ll O\bigl(n^c\bigr)_{c > 1} \ll O(c^n) \ll O(n!) \ll O(n^n).$
  \end{minipage}
  } 