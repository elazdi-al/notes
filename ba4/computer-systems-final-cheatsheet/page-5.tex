\hspace{-20px}
\begin{minipage}[htp]{0.495\textwidth}
    \fbox{\begin{minipage}[htp]{0.99\textwidth}
    \noindent\textbf{Network Addressing \& Web Protocols:}\\
       - \textbf{IP Address:} 32-bit (IPv4) or 128-bit (IPv6) unique identifier for network devices. Format: dotted decimal (e.g., 192.168.1.1) or hexadecimal notation.\\
       - \textbf{Port Number:} 16-bit identifier (0-65535) for specific application/service on a host. Well-known ports: HTTP=80, HTTPS=443, FTP=21, SSH=22.
    \end{minipage}}\\[5px]
    \fbox{\begin{minipage}[htp]{0.99\textwidth}
    \noindent\textbf{HTTP(HyperText Transfer Protocol):} Stateless application-layer protocol for web communication.
    \begin{itemize}[noitemsep,nolistsep,topsep=0px,partopsep=0pt,parsep=0pt]
        \item[-] \textbf{Stateless Protocol:} Server doesn't maintain client state between requests. Each request is independent and self-contained.
        \item[-] \textbf{Web Objects:} Files identified by URLs (HTML pages, images, videos, CSS, JavaScript, etc.). Each object has a unique URL.
        \item[-] \textbf{Web Page:} Base HTML file + referenced objects (images, stylesheets, scripts). \textbf{Single page may require multiple HTTP requests.}
        \item[]  \begin{minipage}[htp]{0.5\textwidth}
            \textbf{HTTP Methods:}\\
 \textbf{GET:} Retrieve resource from server. Safe, idempotent.\\
\textbf{HEAD:} Like GET but returns only headers (no body). Used for metadata.\\
 \textbf{POST:} Submit data to server (forms, uploads). Non-idempotent.\\
 \textbf{PUT:} Update/create resource. Idempotent.\\
 \textbf{DELETE:} Remove resource. Idempotent.
        \end{minipage}
        \hfill
        \begin{minipage}[htp]{0.48\textwidth}
        \textbf{Common Response Codes:}\\
 \textbf{200 OK:} Request successful\\
 \textbf{301/302:} Redirect (permanent/temporary)\\
 \textbf{404 Not Found:} Resource doesn't exist\\
 \textbf{500 Internal Server Error:} Server-side error
        \end{minipage}
        \item[-] \textbf{Cookies:} Small data stored by browser, sent with requests to maintain state across stateless HTTP. Used for authentication, preferences, tracking.
        \item[-] \textbf{Web Caching:} Store frequently accessed objects closer to client to reduce latency and server load. Types: browser cache, proxy cache, and \textbf{Edge Caches} (part of a CDN) which are geographically distributed.
        \item[-] \textbf{Max-Age:} Cache-Control directive specifying how long (in seconds) a resource can be cached before becoming stale. Example: \texttt{max-age=3600} (1 hour).
    \end{itemize} 
    \end{minipage}}\\[5px]
    \fbox{
    \begin{minipage}[htp]{0.99\textwidth}
    \noindent\textbf{DNS (Domain Name System):} Hierarchical distributed naming system that translates human-readable domain names to IP addresses.
    \begin{itemize}[noitemsep,nolistsep,topsep=0px,partopsep=0pt,parsep=0pt]
        \item[-] \textbf{DNS Name:} Hierarchical structure (e.g., www.epfl.ch). Read right-to-left: .ch (TLD), epfl (2nd level), www (subdomain).
        \item[-] \textbf{Reserved Port:} UDP/TCP port 53. 
        \item[-] \textbf{DNS Server Hierarchy:}
        \begin{itemize}[noitemsep,nolistsep,topsep=0px,partopsep=0pt,parsep=0pt]
            \item \textbf{Root DNS Servers:} 13 logical servers (A-M) managing top-level domains
            \item \textbf{TLD DNS Servers:} Manage specific top-level domains (.com, .org, .ch, etc.)
            \item \textbf{Authoritative DNS Servers:} Store actual DNS records for specific domains
        \end{itemize}
        \item[-] \textbf{TTL (Time To Live):} Specifies how long DNS records can be cached (in seconds).
                \item[-] \textbf{Protocol:} Primarily UDP for speed. TCP used for zone transfers or when response exceeds 512 bytes.
        \item[-] \textbf{Client-side Syscalls (UDP):}
        \begin{itemize}[noitemsep,nolistsep,topsep=0px,partopsep=0pt,parsep=0pt]
            \item \textbf{socket():} Creates a new UDP socket endpoint.
            \item \textbf{bind():} (Optional) Assigns a local port to the socket. If not called, an ephemeral port is assigned by the kernel on first `sendto()`.
            \item \textbf{sendto():} Sends the DNS query to the server's IP and port 53.
            \item \textbf{recvfrom():} Waits to receive the DNS response from the server.
        \end{itemize}
    \end{itemize}

    \noindent\textbf{DNS Resolution Approaches:}\\
    \begin{minipage}[htp]{0.48\textwidth}
        \textbf{Iterative Query:}
        \begin{enumerate}[noitemsep,nolistsep,topsep=0px,partopsep=0pt,parsep=0pt]
            \item Client queries local DNS
            \item Local DNS queries root, gets TLD referral
            \item Local DNS queries TLD, gets authoritative referral  
            \item Local DNS queries authoritative, gets answer
            \item Each server returns \textbf{referral} or answer
        \end{enumerate}
        \includegraphics[width=0.9\textwidth]{images/iterative.png}
    \end{minipage}
    \hfill
    \begin{minipage}[htp]{0.48\textwidth}
        \textbf{Recursive Query:}
        \begin{enumerate}[noitemsep,nolistsep,topsep=0px,partopsep=0pt,parsep=0pt]
            \item Client queries local DNS
            \item Local DNS recursively queries on behalf of client
            \item Each server in chain does the work
            \item Final answer returned through chain
            \item Client only sees final result
        \end{enumerate}
        \includegraphics[width=0.9\textwidth]{images/recursive.png}
    \end{minipage}
\end{minipage}}\\[5px]
\noindent\textbf{Packet Switching \& Network Delays:}
\begin{itemize}[noitemsep,nolistsep,topsep=0px,partopsep=0pt,parsep=0pt]
    \item[-] \textbf{Packet Switch:} Network device that forwards packets using store-and-forward switching. Receives entire packet before forwarding.
    \item[-] \textbf{Transmission Delay:} Time to push packet onto link. $\mathbf{d_{trans} = \frac{L}{R}}$, where L is the packet size (bits), R is the transmission rate (bps).
    \item[-] \textbf{Propagation Delay:} Time for signal to travel link. $\mathbf{d_{prop} = \frac{d}{s}}$, where d is the distance, s is the signal speed.
    \item[-] \textbf{Queuing Delay:} Time packet waits in output queue before transmission. Depends on congestion level.
    \item[-] \textbf{Total Packet Delay:} $\mathbf{d_{total} = d_{proc} + d_{queue} + d_{trans} + d_{prop}}$, where $\mathbf{d_{proc}}$ is the processing delay.
    \item[-] \textbf{Packet Loss:} Occurs when packet arrives at full queue buffer. Packet is dropped.
    \item[-] \textbf{Traffic Analysis:}
    \begin{itemize}[noitemsep,nolistsep,topsep=0px,partopsep=0pt,parsep=0pt]
        \item \textbf{Arrival Rate ($\lambda$):} Average rate at which packets arrive.
        \item \textbf{Packet Inter-arrival Time:} Time between consecutive packet arrivals. Average inter-arrival time = $\mathbf{\frac{1}{\lambda}}$.
        \item \textbf{Departure Rate ($\mu$):} Rate at which packets are served/transmitted.
        \item \textbf{If $\mathbf{\lambda > \mu}$:} Queue grows indefinitely, delay increases, eventually packet loss.
        \item \textbf{If $\mathbf{\lambda < \mu}$:} Stable system, finite queuing delay.
        \item \textbf{If $\mathbf{\lambda = \mu}$:} System at capacity, high sensitivity to bursts.
    \end{itemize}
    \item[-] \textbf{Average Throughput:} $\mathbf{\text{min(transmission rates along path)}}$ - the bottleneck link determines end-to-end throughput.
    \item[-] \textbf{Transfer Time (N packets through bottleneck):} Total time to send N packets of size L bits through a path with bottleneck link rate $R_b$.
    \begin{itemize}[noitemsep,nolistsep,topsep=0px,partopsep=0pt,parsep=0pt]
        \item $\mathbf{\text{Transfer Time} = d_{\text{to bottleneck}} + \frac{N \times L}{R_b} + d_{\text{from bottleneck}}}$
        \item Where $d_{\text{to bottleneck}}$ and $d_{\text{from bottleneck}}$ include propagation and transmission delays on non-bottleneck links
    \end{itemize}
\end{itemize}
\textbf{Congestion Control:} A network-wide mechanism where senders reduce their transmission rate to prevent overwhelming the network's capacity.\\
\textbf{Flow Control:} A point-to-point mechanism where the receiver informs the sender how much data it can handle, preventing the sender from overwhelming the receiver's buffer.
\end{minipage}
\hfill
\begin{minipage}[htp]{0.495\textwidth}
    \fbox{\begin{minipage}[htp]{0.99\textwidth}
    \noindent\textbf{Internet Structure:}
\begin{itemize}[noitemsep,nolistsep,topsep=0px,partopsep=0pt,parsep=0pt]
    \item[-] \textbf{ISP (Internet Service Provider):} Company that provides Internet access.
    \begin{itemize}[noitemsep,nolistsep,topsep=0px,partopsep=0pt,parsep=0pt]
        \item \textbf{Tier-1 ISP:} Global network, peers with all other Tier-1s. The backbone.
        \item \textbf{Regional ISP:} Connects to one or more Tier-1 ISPs.
        \item \textbf{Access ISP (Last-mile):} Connects end-users (homes, businesses).
    \end{itemize}
    \item[-] \textbf{IXP (Internet Exchange Point):} Physical infrastructure where multiple ISPs connect to exchange traffic directly, avoiding Tier-1 transit.
\end{itemize}
\end{minipage}}\\[5px]
\fbox{\begin{minipage}[htp]{0.99\textwidth}
    \noindent\textbf{UDP (User Datagram Protocol):} Connectionless, unreliable transport protocol.
    \begin{itemize}[noitemsep,nolistsep,topsep=0px,partopsep=0pt,parsep=0pt]
        \item[-] \textbf{Characteristics:} No connection establishment, no reliability guarantees, no flow/congestion control, minimal overhead.
        \item[-] \textbf{Header Structure (8 bytes total):}
        \begin{itemize}[noitemsep,nolistsep,topsep=-5px,partopsep=0pt,parsep=0pt]
            \item Source Port: 16 bits
            \item Destination Port: 16 bits  
            \item Length: 16 bits (UDP header + data length)
            \item Checksum: 16 bits (optional in IPv4, mandatory in IPv6)
        \end{itemize}
    \end{itemize}
    \end{minipage}}\\[5px]
\fbox{\begin{minipage}[htp]{0.99\textwidth}
    \noindent\textbf{TCP Protocol Essentials.} Transmission Control Protocol provides reliable, ordered delivery over unreliable networks.
    \begin{itemize}[noitemsep,nolistsep,topsep=0px,partopsep=0pt,parsep=0pt]
        \item[-] \textbf{Three-Way Handshake:} Connection establishment process.
        \begin{itemize}[noitemsep,nolistsep,topsep=-5px,partopsep=0pt,parsep=0pt]
            \item \textbf{Step 1:} Client sends SYN with initial sequence number (ISN)
            \item \textbf{Step 2:} Server responds with SYN-ACK (server ISN + ACK of client ISN+1)
            \item \textbf{Step 3:} Client sends ACK (acknowledges server ISN+1)
        \end{itemize}
        \item[-] \textbf{Sequence Number (SEQ):} 32-bit field identifying byte position in data stream. Initial value chosen randomly. Each byte of data increments SEQ by 1. Used for ordering and duplicate detection.
        \item[-] \textbf{Acknowledgment Number (ACK):} 32-bit field indicating next expected sequence number from sender.
    \end{itemize}
\end{minipage}}    
 \textbf{Congestion Window (cwnd):} TCP sender's estimate of how many bytes can be sent without causing network congestion, dynamically adjusted based on network conditions.
        \begin{itemize}[noitemsep,nolistsep,topsep=0px,partopsep=0pt,parsep=0pt]
            \item \textbf{Purpose:} Controls sending rate to prevent network overload and packet loss due to buffer overflow at intermediate routers.
            \item \textbf{Relationship:} $\mathbf{\text{Effective Window} = \min(\text{cwnd}, \text{rwnd})}$ where rwnd = receiver advertised window.
        \end{itemize}
 \textbf{Maximum Segment Size (MSS):} Largest amount of data bytes that TCP can send in a single segment, excluding TCP and IP headers.\\
 \textbf{Timeout Calculation:} TCP adaptively calculates retransmission timeout to balance quick recovery from lost packets with avoiding unnecessary retransmissions due to network delay variations.
        \begin{itemize}[noitemsep,nolistsep,topsep=-5px,partopsep=0pt,parsep=0pt]
            \item \textbf{SampleRTT:} Measured time for a specific segment to be sent and acknowledged (not retransmitted segments)
            \item \textbf{EstimatedRTT:} Exponentially weighted moving average of RTT samples to smooth out fluctuations\\
            $\mathbf{\text{EstimatedRTT} = (1-\alpha) \times \text{EstimatedRTT} + \alpha \times \text{SampleRTT}}$\\where $\mathbf{\alpha}$ = smoothing factor for RTT estimation (typically 0.125)\\

            \item \textbf{DevRTT:} Estimate of RTT variation to account for network jitter and delay variability\\
            $\mathbf{\text{DevRTT} = (1-\beta) \times \text{DevRTT} + \beta \times |\text{SampleRTT} - \text{EstimatedRTT}||}$\\where $\mathbf{\beta}$ = smoothing factor for deviation estimation (typically 0.25)\\
        \end{itemize}
$\mathbf{\text{TimeoutInterval} = \text{EstimatedRTT} + 4 \times \text{DevRTT}}$\\
\textbf{Checksum:} 16-bit Internet checksum. Computed over TCP header, data, and pseudo-header (source IP, dest IP, protocol, TCP length). Uses 1's complement arithmetic.\\[-5px]
\hrule
\vspace*{2px}
 \textbf{TCP Congestion Control (Tahoe \& Reno):} Algorithms that dynamically adjust sending rate to prevent network congestion while maximizing throughput.
        \begin{itemize}[noitemsep,nolistsep,topsep=0px,partopsep=0pt,parsep=0pt]
            \item \textbf{Common Phases (Both Algorithms):}
            \begin{itemize}[noitemsep,nolistsep,topsep=-5px,partopsep=0pt,parsep=0pt]
                \item \textbf{Slow Start:} $\mathbf{\text{cwnd} = \text{cwnd} + \text{MSS}}$ for each ACK. Exponential growth until cwnd $\geq$ ssthresh.
                \item \textbf{Congestion Avoidance:} $\mathbf{\text{cwnd} = \text{cwnd} + \frac{\text{MSS}^2}{\text{cwnd}}}$ for each ACK. Linear growth.
                \item \textbf{Fast Retransmit vs No Fast Retransmit:}
                \begin{itemize}[noitemsep,nolistsep,topsep=-5px,partopsep=0pt,parsep=0pt]
                    \item \textbf{Enabled:} Upon 3 duplicate ACKs, immediately retransmit lost segment. Faster recovery, reduced latency.
                    \item \textbf{Disabled:} Must wait for timeout expiration to detect and retransmit lost segments. Slower recovery, higher latency.
                \end{itemize}
            \end{itemize}
      \end{itemize}
      \vspace*{5px}
      \fbox{\begin{minipage}[htp]{0.99\textwidth}
        \begin{minipage}[htp]{0.35\textwidth}
            \textbf{Tahoe \\No fast-retransmit (Conservative):}\\
            \vfill
           \textbf{Timeout}\\   ssthresh = $\mathbf{\lfloor\text{cwnd}/2\rfloor}$, \\cwnd = MSS (slow start)\\
           \textbf{Behavior:}\\ Always returns to slow start after ANY loss (more conservative).\\
           
        \end{minipage}   
        \hfill
        \begin{minipage}[htp]{0.65\textwidth}
        \includegraphics[width=1\textwidth]{images/tahoe.png}
        \end{minipage}
    \end{minipage}}\\[5px]
    \fbox{\begin{minipage}[htp]{0.99\textwidth}
        \begin{minipage}[htp]{0.35\textwidth}
        \textbf{Reno (Adaptive):}\\
        \vfill
           \textbf{Timeout:}\\ssthresh = $\mathbf{\lfloor\text{cwnd}/2\rfloor}$, \\cwnd = MSS (slow start)\\
           \textbf{3 Dup ACKs:} \\ssthresh = $\mathbf{\lfloor\text{cwnd}/2\rfloor}$, \\cwnd = ssthresh + 3×MSS \\(fast recovery)\\
        \end{minipage}
        \hfill
        \begin{minipage}[htp]{0.65\textwidth}
            \includegraphics[width=1\textwidth]{images/reno.png}
        \end{minipage}
    \end{minipage}}
 \vspace*{35px}
\end{minipage}

